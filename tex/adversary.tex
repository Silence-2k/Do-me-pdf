% --- estilos.tex ---
\usepackage{xcolor}
\usepackage{enumitem}
\usepackage{tcolorbox}
\tcbuselibrary{skins}

% Colores
\definecolor{bordercolor}{HTML}{bcab84}
\definecolor{bgcolor}{HTML}{f4f0e5}
\definecolor{highlight}{HTML}{bcab84}

% Caja principal
\newtcolorbox{adversarybox}[1]{
    enhanced,
    colback=gray!10,
    colframe=gray!70,
    arc=5pt,
    boxrule=1pt,
    outer arc=5pt,
    after skip=1em,
    top=8pt,
    left=8pt,
    right=8pt,
    bottom=8pt,
    before upper={\parskip=4pt},
}

% Caja interna
\newtcolorbox{adversaryinnerbox}{
    colback=white,
    enhanced,
    frame hidden,
    borderline north={1pt}{0pt}{gray!70},
    borderline south={1pt}{0pt}{gray!70},
    left=0.5em,
    right=0.5em,
    top=0.5em,
    bottom=0.5em,
    before skip=1em,
    after skip=1em,
    sharp corners,
}

% Separador
\newcommand{\myseparator}{\par\noindent\rule{\linewidth}{1pt}\par}

\newcommand{\adversary}[6]{
    \begin{adversarybox}{}
        \eveleth\fontsize{12pt}{12pt}\selectfont\MakeTextUppercase{#1}
        \par\smallskip

        \normalfont\fontsize{8pt}{8pt}\selectfont
        \textbf{\emph{#2}}
        \par\smallskip

        \emph{#3}
        \par\smallskip

        \textbf{Motives \& Tactics:} #4
        \par\smallskip

        \begin{adversaryinnerbox}
            #5
        \end{adversaryinnerbox}

        #6
    \end{adversarybox}
}